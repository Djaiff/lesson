\documentclass[a4paper,10pt,french]{article}

\usepackage[utf8]{inputenc}
\usepackage[T1]{fontenc} 
\usepackage{graphicx,psfrag}
\usepackage[frenchb]{babel}
\usepackage{ae,aecompl}
\usepackage{float}
\usepackage{fancyhdr}
\renewcommand{\baselinestretch}{1.2}

%%%%%%%%%%%%%%%%%%%%%%%%%%%%%%%%%%%%%%%%%%%%%%%%%%%%%%%%%%%%

\setlength{\textwidth}{18cm}
\setlength{\textheight}{25cm}
\setlength{\oddsidemargin}{-1cm}
\setlength{\evensidemargin}{0cm}
\setlength{\topmargin}{-2cm}
\setlength{\parindent}{0cm}

%%%%%%%%%%%%%%%%%%%%%%%%%%%%%%%%%%%%%%%%%%%%%%%%%%%%%%%%%%%%
\pagestyle{fancy}

\renewcommand{\footrulewidth}{0.4 pt}
\renewcommand{\headrulewidth}{0 pt}

\newcommand{\fct}[1]{\texttt{#1}}
\newcommand{\menu}[1]{\texttt{#1}}
\newcommand{\code}[1]{\texttt{#1}}
\newcommand{\pkg}[1]{\textsf{#1}}

%%%%%%%%%%%%%%%%%%%%%%%%%%%%%%%%%%%%%%%%%%%%%%%%%%%%%%%%%%%%

\begin{document}
\thispagestyle{empty}

%\vspace*{-3cm}~\\
\hspace*{-0.5cm}

%%%%%%%%%%%%%%%%%%%%%%%%%%%% Titre %%%%%%%%%%%%%%%%%%%%%%%%%%%%%%%%%%%%
\begin{center}
\LARGE Introduction à \textsf{R}
\end{center}
\bigskip

%%%%%%%%%%%%%%%%%%%%%%%%%%%%%%%%%%%%%%%%%%%%%%%%%%%%%%%%%%%%%%%%%%%%%%
\section{Entrer, sortir et trouver de l'aide avec \textsf{R}}
\textsf{R} ouvre une session à chaque entrée. Par defaut (sous windows)
la session est toujours dans le même répertoire. Pour changer de
répertoire utiliser le menu.

Pour avoir de l'aide sur la fonction \texttt{mean}, il suffit de taper
\begin{verbatim}
> help(mean)
\end{verbatim}

Une aide html est disponible avec 
\begin{verbatim}
> help.start()
\end{verbatim}

A la sortie, la session peut être sauvegardée, ce qui en général n'est
pas très utile.
\section{\textsf{R} est une (super) machine à calculer}
\subsection{Calculs simples}
Exécutez et commentez :
\begin{verbatim}
> 2+2
> # blabla
> blabla
> 2+2 # ceci est une addition
> pi 
> exp(2)
> log(10)
> sin(5*pi)
> (1+3/5)*5
\end{verbatim}
\subsection{Calculs sur plusieurs valeurs}
Si nous voulons faire une moyenne de notes, il faut pouvoir manipuler
plusieurs valeurs ensemble (donc un vecteur). Effectuons la moyenne
(\texttt{mean}) de 4, 10 et 16:
\begin{verbatim}
> mean(c(4,10,16))
\end{verbatim}
En utilisant la ligne suivante
\begin{verbatim}
> c(4,10,16)
\end{verbatim}
trouvez à quoi sert la fonction \code{c()}.

\medskip
\textbf{Exercices d'application:}
\begin{enumerate}
\item Calculer la moyenne de 1, 3, 5, 4, et 8.
\item Calculer la somme (\texttt{sum}) de 4, 10 et 16
\item Calculer la médiane  (\texttt{median}) de 4, 10 et 16
\end{enumerate}

\subsection{Mettre en mémoire plusieurs valeurs}
Nous souhaitons stocker un vecteur pour le réutiliser. Nous devons donc \emph{affecter} des valeurs à un nom. Executer et commenter ces ordres:
\begin{verbatim}
> x <- pi
> print(x)
> x
> objects()
> y=pi
> objects()
> y
> x <- c(4,10,16) 
> print(x)
> x
\end{verbatim}
Conclusion: l'affectation crée l'objet ou écrase l'objet. L'affectation
est réalisée par \verb+<-+ ou \verb+=+.

\medskip
\textbf{Exercices d'application:}
\begin{enumerate}
\item Calculer le max (\texttt{max}) de \texttt{x}.
\item Calculer le min (\texttt{min}) de \texttt{x}.
\item Calculer la moyenne (\texttt{mean}) de \texttt{x}.
\item Calculer la longueur (\texttt{length}) de \texttt{x}.
\item Calculer le résumé  numérique (\texttt{summary}) de \texttt{x}.
\end{enumerate}

\section{\textsf{R} manipule des vecteurs}
\subsection{Calcul vectoriel}
Additionnons 2 vecteurs:
\begin{verbatim}
> y=c(-1,5,0)
> x
> y
> x+y
> -y
\end{verbatim}
Commenter les deux derniers ordres ci-dessus et ceux ci-dessous
\begin{verbatim}
> x+2
> abs(y)
\end{verbatim}
Intéressons nous à la multiplication. Exécuter et commenter:
\begin{verbatim}
> x*y
> x/y
> x^2
\end{verbatim}
Une nouvelle opération \verb+:+
\begin{verbatim}
> 1:3
> 1:10
> -1:5
> -(1:5)
\end{verbatim}

\subsection{Vecteur de logiques}
Les logiques sont soit \code{TRUE} soit \code{FALSE} (que l'on peut abbrévier par \code{T} et \code{F}\footnote{attention aux majuscules})
\begin{verbatim}
> w <- c(TRUE,FALSE,FALSE)
> sum(w)
> any(w)
> all(w)
> !w
> (TRUE)&(FALSE)
> (TRUE)|(FALSE)
> (TRUE)|(TRUE)
\end{verbatim}

\subsection{Valeurs spéciales et  calculs}
La valeur \code{NA} est la valeur manquante.
La valeur \code{NaN} 
est la valeur \og Not a Number\fg{} (forme indéterminée). Enfin \code{Inf} est l'infini. Ces valeurs
\begin{verbatim}
> log(0)
> log(Inf)
> 1/0
> 0/0
> max(c( 0/0,1,10))
> max(c(NA,1,10))
> max(c(-Inf,1,10))
> is.finite(c(-Inf,1,10))
> is.na(c(NA,1,10))
> is.nan(c(NaN,1,10))
\end{verbatim}

\begin{verbatim}

\end{verbatim}
\subsection{Créer des vecteurs}
\begin{enumerate}
\item Créer le vecteur d'entier de 5 à 23.
\item Créer le vecteur de 6 à 24 allant de 2 en 2.
\item Créer le vecteur de 100 valeurs régulièrement espacées entre 0 et 1.
\item Créer le vecteur suivant
\begin{verbatim}
 [1] 1 2 3 4 5 1 2 3 4 5 1 2 3 4 5 1 2 3 4 5
\end{verbatim}
\item Créer le vecteur suivant
\begin{verbatim}
> [1] 1 1 1 2 2 2 3 3 3 4 4 4 5 5 5
\end{verbatim}
\item Créer le vecteur suivant
\begin{verbatim}
[1] 1 1 2 2 2 3 3 3 3
\end{verbatim}
\end{enumerate}


\subsection{Sélection dans un vecteur}
Sélectionnons les coordonnées dans un vecteur par leur numéro:
\begin{verbatim}
> x[1]
> x[2]
> x[c(1,2,3)]
> x[1:3]
> x[c(2,2,1,3)]
> x[c(1:3,2,1)]
> x[-1]
> x[-c(1,2)]
> x[-(1:2)]
\end{verbatim}

Sélectionnons les coordonnées dans un vecteur par des logiques. Pour cela, commenter les ordres suivants:
\begin{verbatim}
> objects()
> vec1<-c(3,NA,4)
> objects()
> vec2<-c(FALSE,TRUE,FALSE)
> objects(pattern="vec*")
> vec2
> vec1
> vec1[vec2]
> is.na(vec1)
\end{verbatim}

Tapez l'ordre suivant sans essayer de l'interpréter (dans un premier
temps).
\begin{verbatim}
> vec1<-runif(20) ; vec1[vec1>0.5]<-NA
\end{verbatim}
En utilisant le groupe d'ordre précédent, remplacez les valeurs manquantes
de vec1 par 0 ; retournez sur la ligne précédente et l'interpréter.


\subsection{Chaînes de caractères (pour aller plus loin)}
Exécutez et commenter
\begin{verbatim}
> z=c("aze","fds")
> z[1]
> paste("m",1:3)
> paste("m",1:3,sep="")
> c(paste("m",1:3,sep=""),paste("p",1:4,sep="."),z)
\end{verbatim}


\section{\textsf{R} est un (super) tableur}
\begin{enumerate}
\item Importer les données du tableau contenu dans le fichier \texttt{tab1.csv} dans l'objet \textsf{R} que vous appelerez \texttt{don1}. Le resultat de l'affichage à l'écran doit être
\begin{verbatim}
> don1
  V1 V2
1  1  2
2  0  2
3  3  1
\end{verbatim}

\item Importer les données du tableau contenu dans le fichier \texttt{tab2.csv} dans  l'objet \textsf{R} que vous appelerez  \texttt{don2}. Le resultat de l'affichage à l'écran doit être
\begin{verbatim}
> don2
  variable1 variable2
1      -1.0         0
2       2.0        -2
3       3.1         4
\end{verbatim}
\item Importer les données du tableau contenu dans le fichier \texttt{tab3.csv} dans  l'objet \textsf{R} que vous appelerez  \texttt{don3}. Le resultat de l'affichage à l'écran doit être
\begin{verbatim}
> don3
        sexe  taille
  gege  masculin   180.6
simone   feminin   175.2
albert  masculin   172.9
\end{verbatim}

\item Importer les données du tableau contenu dans le fichier \texttt{tournesols.csv} dans  l'objet \textsf{R} que vous appelerez  \texttt{tournesols}. Ce tableau contient les mesures, sur différents individus statistiques (plantes de tournesols sauvages), des variables décrites dans le tableau \ref{tab:variables:plantes}.  
\begin{table}[H]
  \centering
  \begin{tabular}{cl}\hline\hline
Code variable& Descriptif variable\\\hline
    ecotype & code plante \\
plt & numéro du plant d'un écotype donné\\
etat & état d'origine de la plante (aux USA)\\
longitude& longitude du lieu de collecte (aux USA)\\
latitude& latitude du lieu de collecte (aux USA)\\
haut& hauteur des plants\\
semflo& jour de floraison (écart en jour par rapport au premier mai)\\
rambas&note de ramification basale (entre 0 aucune et 4 maximum)\\
longfeu&longueur du cumulée du limbe et du pétiole (cm ?)\\ 
grlon&longueur maxi de la graine (mm, moyenne sur 15 graines minimum)\\
huile&pourcentage d'huile\\\hline
  \end{tabular}
  \caption{Variables mesurées sur les tournesols (dans la station d'essai aux environs de Montpellier).}
  \label{tab:variables:plantes}
\end{table}
Ce tableau sera affecté (après importation dans \textsf{R}) dans l'objet appelé \texttt{tournesols} et l'affichage du résumé de celui-ci à l'écran doit être:
\begin{verbatim}
> summary(tournesols)
    ecotype            plt              etat       longitude      
 Min.   : 209.0   Min.   : 1.000   TEX    : 40   Min.   :-121.77  
 1st Qu.: 511.0   1st Qu.: 2.000   WYO    : 38   1st Qu.:-108.54  
 Median : 775.0   Median : 4.000   COL    : 35   Median :-104.08  
 Mean   : 759.1   Mean   : 3.991   CLF    : 34   Mean   :-104.61  
 3rd Qu.: 975.0   3rd Qu.: 5.000   SDK    : 34   3rd Qu.: -99.85  
 Max.   :1150.0   Max.   :10.000   UTA    : 25   Max.   : -88.07  
                                   (Other):121                    
    latitude         semflo            haut           rambas     
 Min.   :26.19   Min.   : 18.00   Min.   : 50.0   Min.   :0.000  
 1st Qu.:35.00   1st Qu.: 30.00   1st Qu.:145.0   1st Qu.:2.000  
 Median :39.17   Median : 40.00   Median :190.0   Median :3.000  
 Mean   :38.96   Mean   : 40.43   Mean   :186.1   Mean   :2.633  
 3rd Qu.:43.77   3rd Qu.: 47.00   3rd Qu.:227.5   3rd Qu.:3.000  
 Max.   :48.78   Max.   :113.00   Max.   :330.0   Max.   :4.000  
                                                                 
    longfeu          grlon           huile      
 Min.   :10.00   Min.   :3.680   Min.   :15.24  
 1st Qu.:27.00   1st Qu.:4.480   1st Qu.:24.70  
 Median :42.00   Median :4.840   Median :26.69  
 Mean   :39.47   Mean   :5.004   Mean   :26.75  
 3rd Qu.:51.00   3rd Qu.:5.370   3rd Qu.:28.86  
 Max.   :70.00   Max.   :8.160   Max.   :35.50  
\end{verbatim}
\end{enumerate}
\subsection{Les noms des variables (colonnes) et des individus (lignes)}
Exécuter et commenter:
\begin{verbatim}
> rownames(don3) ## c'est un vecteur
> names(don3)    ## c'est un vecteur
> colnames(don3) ## c'est un vecteur
> colnames(don1)[2]
> colnames(don1)[2] <- "var2"
> colnames(don1)
> colnames(don1) <- colnames(don2)
> colnames(don1) <- c("variable1","variable2")
\end{verbatim}

\subsection{Sélection dans des tableaux}
Exécuter et commenter:
\begin{verbatim}
> don1[1,]
> don3[,"sexe"]
> don3$sexe
> don3[,2]
> don3[,c(FALSE,TRUE)]
> don3[,c("taille","sexe")]
> don1[1,2]
> don1[,1:2]
> don1[-1,]
> don1[c(2,3),c(2,1)]
> don1[c(TRUE,FALSE,TRUE),]
> don1[don1[,1]>0,]
> don1[-(1),]
> don1[,c(2,1)]
\end{verbatim}
\subsection{Opération sur les colonnes}
Exécuter et commenter:
\begin{verbatim}
> don1[,1]+don1[,2]
> exp(don1[,1])
> don3[1,]+don3[2,]
\end{verbatim}
\subsection{\textsf{R} et résumé de tableaux}
Exécuter:
\begin{verbatim}
> summary(tournesols) ## tableau issu de l'importation de tournesol.csv
\end{verbatim}
Quels sont les types de chaque variable (quantitatif ou qualitatif) ?
\section{Elimination des valeurs manquantes}
Importez le tableau \texttt{tournesols\string_brut.csv}.

Lorsque l'on souhaite connaître les valeurs manquantes (\code{NA}) d'un vecteur:
\begin{verbatim}
> which(is.na(tbrut[,"huile"])
\end{verbatim}
Lorsque l'on souhaite connaître les valeurs manquantes (\code{NA}) de tout un tableau:
\begin{verbatim}
> which(is.na(tbrut),arr.ind=TRUE)
\end{verbatim}
\begin{enumerate}
\item Affichez (grâce à une fonction)  les noms de la première colonne et de la
  seconde colonne du tableau résultant de l'ordre précédent (\code{NA} dans tout un tableau).
\item Affichez la première ligne. Que représente t-elle?
\item En utilisant la réponse à la question précédente, enlever les
  lignes qui comportent des valeurs manquantes. Affecter le résultat
  dans le tableau \texttt{tbrut2}.
\end{enumerate}

\end{document}